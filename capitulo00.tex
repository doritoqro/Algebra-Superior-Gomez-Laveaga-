\chapter{Algo de lógica}

\setcounter{section}{3}
\section{Ejercicios del capítulo 0}

\begin{enumerate}[label=0.1.\arabic*]
	% Inciso --- 0.1.1 ---
	\item Determine cuáles de las siguientes oraciones son proposiciones.
	\begin{enumerate}[label=(\arabic*)]
		% Inciso --- (1) ---
		\item El 7 de diciembre de 1941 fue domingo. \\
		\solucion: \\
		
		% Inciso --- (2) ---
		\item Algunos números enteros son negativos. \\
		\solucion: \\
		
		% Inciso --- (3) ---
		\item ¡Si todas las mañanas fueran tan soleadas y despejadas como ésta! \\
		\solucion: \\
		
		% Inciso --- (4) ---
		\item El número 15 es un número par. \\
		\solucion: \\
		
		% Inciso --- (5) ---
		\item Esta frase es falsa. \\
		\solucion: \\
		
		% Inciso --- (6) ---
		\item ¿Qué hora es? \\
		\solucion: \\
		
		% Inciso --- (7) ---
		\item Todos los círculos del mismo radio son iguales. \\
		\solucion: \\
		
		% Inciso --- (8) ---
		\item En los números enteros, $ 11+6 \neq 12 $. \\
		\solucion: \\
		
		% Inciso --- (9) ---
		\item La tierra es redonda. \\
		\solucion: \\
		
	\end{enumerate}

	% Inciso --- 0.1.2 ---
	\item Diga si cada una de las siguientes proposiciones acerca de los números enteros es verdadera o falsa.
	\begin{enumerate}[label=(\arabic*)]
		% Inciso --- (1) ---
		\item $ (3+1 = 4) \vee (2+5 = 9) $ . \\
		\solucion: \\
		
		% Inciso --- (2) ---
		\item $ (5-1 = 4) \wedge (9+12 \neq 7) $ . \\
		\solucion: \\
		
		% Inciso --- (3) ---
		\item $ (3<10 = 4) \vee (7 \neq 2) $ . \\
		\solucion: \\
		
		% Inciso --- (4) ---
		\item $ (4 = 11-7) \Longrightarrow (8>10) $ . \\
		\solucion: \\
		
		% Inciso --- (5) ---
		\item $ (4^2 \neq 16) \Longrightarrow (4-4 = 8) $ . \\
		\solucion: \\
		
		% Inciso --- (6) ---
		\item $ (5+2 = 10) \Longleftrightarrow (17+19=36) $ . \\
		\solucion: \\
		
		% Inciso --- (7) ---
		\item $ (6 = 5) \Longleftrightarrow (12 \neq 12) $ . \\
		\solucion: \\
		
	\end{enumerate}

	% Inciso --- 0.1.3 ---
	\item Comparar las tablas de verdad de $ \neg (P \vee Q) $ y $ (\neg P) \vee Q $. \\
	\solucion: \\
	
	% Inciso --- 0.1.4 ---
	\item Sean $ P $ y $ Q $ proposiciones tales que $ P \Longrightarrow Q $ es falsa. Determine los valores de verdad de
	\begin{enumerate}[label=(\arabic*)]
		% Inciso --- (1) ---
		\item $ \neg P \vee Q $ \\
		\solucion: \\
		
		% Inciso --- (2) ---
		\item $ P \wedge Q $ \\
		\solucion: \\
		
		% Inciso --- (3) ---
		\item $ Q \Longrightarrow P $ \\
		\solucion: \\
		
		% Inciso --- (4) ---
		\item $ \neg Q \Longrightarrow \neg P $ \\
		\solucion: \\
		
		% Inciso --- (5) ---
		\item $ P \Longleftrightarrow Q $ \\
		\solucion: \\
		
	\end{enumerate}

	% Inciso --- 0.1.5 ---
	\item Si $ P $ y $ R $ representan proposiciones verdaderas y $ Q $ y $ S $ representan proposiciones falsas, encuentre el valor de verdad de las proposiciones compuestas dadas a continuación:
	\begin{enumerate}[label=(\arabic*)]
		% Inciso --- (1) ---
		\item $ \neg P \wedge R $ \\
		\solucion: \\
		
		% Inciso --- (2) ---
		\item $ \neg Q \vee \neg R $ \\
		\solucion: \\
		
		% Inciso --- (3) ---
		\item $ \neg [ \neg P \wedge (\neg Q \vee P) ] $ \\
		\solucion: \\
		
		% Inciso --- (4) ---
		\item $ \neg [ ( \neg P \wedge \neg Q ) \vee \neg Q ] $ \\
		\solucion: \\
		
		% Inciso --- (5) ---
		\item $ (P \wedge R) \vee \neg Q $ \\
		\solucion: \\
		
		% Inciso --- (6) ---
		\item $ (Q \vee \neg R) \wedge P $ \\
		\solucion: \\
		
		% Inciso --- (7) ---
		\item $ ( \neg P \wedge Q ) \vee \neg R $ \\
		\solucion: \\
		
		% Inciso --- (8) ---
		\item $ \neg ( P \wedge Q ) \wedge (P \wedge \neg Q) $ \\
		\solucion: \\
		
		% Inciso --- (9) ---
		\item $ (\neg R \wedge \neg Q) \vee (\neg R \wedge Q) $ \\
		\solucion: \\
		
		% Inciso --- (10) ---
		\item $ \neg [ ( \neg P \wedge Q ) \vee R ] $ \\
		\solucion: \\
		
		% Inciso --- (11) ---
		\item $ \neg [ R \vee ( \neg Q \wedge \neg P ) ] $ \\
		\solucion: \\
		
		% Inciso --- (12) ---
		\item $ \neg P \Longrightarrow \neg Q $ \\
		\solucion: \\
		
		% Inciso --- (13) ---
		\item $ \neg (P \Longrightarrow Q) $ \\
		\solucion: \\
		
		% Inciso --- (14) ---
		\item $ ( P \Longrightarrow Q ) \Longrightarrow R $ \\
		\solucion: \\
		
		% Inciso --- (15) ---
		\item $ P \Longrightarrow ( Q \Longrightarrow R ) $ \\
		\solucion: \\
		
		% Inciso --- (16) ---
		\item $ [S \Longrightarrow (P \wedge \neg R)] \wedge [( P \Longrightarrow (R \vee Q) ) \wedge S] $ \\
		\solucion: \\
		
		% Inciso --- (17) ---
		\item $ [(P \vee \neg Q) \Longrightarrow (Q \wedge R)] \Longrightarrow (S \vee \neg Q) $ \\
		\solucion: \\
		
	\end{enumerate}

	% Inciso --- 0.1.6 ---
	\item
	\begin{enumerate}[label=(\alph*)]
		% Inciso --- (a) ---
		\item Si la proposición $ Q $ es verdadera, determine todas las asignaciones de valores de verdad para las proposiciones $ P $, $ R $ y $ S $ para que la proposición
		\[ [Q \Longrightarrow ((\neg P \vee R) \wedge \neg S)] \wedge [\neg S \Longrightarrow (\neg R \wedge Q)] \]
		sea verdadera. \\
		\solucion: \\
		
		% Inciso --- (b) ---
		\item Responda la parte (a) si $ Q $ es falsa. \\
		\solucion: \\
		
	\end{enumerate}

	
\end{enumerate}


