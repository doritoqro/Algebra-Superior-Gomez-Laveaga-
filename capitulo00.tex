\chapter{Algo de lógica}

\setcounter{section}{3}
\section{Ejercicios del capítulo 0}

\begin{enumerate}[label=0.1.\arabic*]
	\item Determine cuáles de las siguientes oraciones son proposiciones.
	\begin{enumerate}[label=(\arabic*)]
		\item El 7 de diciembre de 1941 fue domingo. \\
		\solucion: \\
		
		\item Algunos números enteros son negativos. \\
		\solucion: \\
		
		\item ¡Si todas las mañanas fueran tan soleadas y despejadas como ésta! \\
		\solucion: \\
		
		\item El número 15 es un número par. \\
		\solucion: \\
		
		\item Esta frase es falsa. \\
		\solucion: \\
		
		\item ¿Qué hora es? \\
		\solucion: \\
		
		\item Todos los círculos del mismo radio son iguales. \\
		\solucion: \\
		
		\item En los números enteros, $ 11+6 \neq 12 $. \\
		\solucion: \\
		
		\item La tierra es redonda. \\
		\solucion: \\
		
	\end{enumerate}

	\item Diga si cada una de las siguientes proposiciones acerca de los números enteros es verdadera o falsa.
	\begin{enumerate}[label=(\arabic*)]
		\item $ (3+1 = 4) \vee (2+5 = 9) $ . \\
		\solucion: \\
		
		\item $ (5-1 = 4) \wedge (9+12 \neq 7) $ . \\
		\solucion: \\
		
		\item $ (3<10 = 4) \vee (7 \neq 2) $ . \\
		\solucion: \\
		
		\item $ (4 = 11-7) \Longrightarrow (8>10) $ . \\
		\solucion: \\
		
		\item $ (4^2 \neq 16) \Longrightarrow (4-4 = 8) $ . \\
		\solucion: \\
		
		\item $ (5+2 = 10) \Longleftrightarrow (17+19=36) $ . \\
		\solucion: \\
		
		\item $ (6 = 5) \Longleftrightarrow (12 \neq 12) $ . \\
		\solucion: \\
		
	\end{enumerate}

	\item Comparar las tablas de verdad de $ \neg (P \vee Q) $ y $ (\neg P) \vee Q $. \\
	\solucion: \\
	
\end{enumerate}


