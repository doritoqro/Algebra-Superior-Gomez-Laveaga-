\chapter{Conjuntos, relación y funciones}

\setcounter{section}{8}
\section{Ejercicios sección 1.1}

\begin{enumerate}[label=1.1.\arabic*.]
	% Inciso --- 1.1.1 ---
	\item Demuestre el teorema 1.1.11. (pág. 22.) \\
	\solucion: \\
	
	% Inciso --- 1.1.2 ---
	\item Encuentre una proposición adecuada para describir a cada uno de los siguientes conjuntos.
	\begin{enumerate}[label=(\arabic*)]
		% Inciso --- (1) ---
		\item $ A = \{ 0,2,4,6,8,10 \} $ \\
		\solucion: \\
		
		% Inciso --- (2) ---
		\item $ B = \{ 1,3,5,7,9,11 \} $ \\
		\solucion: \\
		
		% Inciso --- (3) ---
		\item $ C = \{ 30,31,32, \ldots \} $ \\
		\solucion: \\
		
		% Inciso --- (4) ---
		\item $ D = \{ 1,4,9,16,25,36,\ldots \} $ \\
		\solucion: \\
		
		% Inciso --- (5) ---
		\item $ E = \{ -1, 2, -3, 4, -5, 6, -7, \ldots \} $ \\
		\solucion: \\
		
		% Inciso --- (6) ---
		\item $ F = \{ -1, 3, -5, 7, -9, 11, \ldots \} $ \\
		\solucion: \\
		
		% Inciso --- (7) ---
		\item $ G = \{ \frac{1}{3}, \frac{1}{4}, \frac{1}{5}, \frac{1}{6}, \ldots \} $ \\
		\solucion: \\
		
	\end{enumerate}

	% Inciso --- 1.1.3 ---
	\item Describa los siguientes conjuntos listando todos sus elementos.
	\begin{enumerate}[label=(\arabic*)]
		% Inciso --- (1) ---
		\item $ \{ x \in \mathbb{Z} \ | \ x^2 + x = 6 \} $ \\
		\solucion: \\
		
		% Inciso --- (2) ---
		\item $ \{ n + \frac{1}{n} \ | \ n \in \{ 1, 2, 3, 5, 7 \} \} $ \\
		\solucion: \\
		
		% Inciso --- (3) ---
		\item $ \{ x \in \mathbb{N} \ | \ x \ \text{ es número par y } \ x^2 \leq 50 \} $ \\
		\solucion: \\
		
		% Inciso --- (4) ---
		\item $ \{ 1 + (-1)^n \ | \ n \in \mathbb{N} \} $ \\
		\solucion: \\
		
		% Inciso --- (5) ---
		\item $ \{ x \in \mathbb{N} \ | \ x^2 - 3x = 0 \} $ \\
		\solucion: \\
		
		% Inciso --- (6) ---
		\item $ \{ n^3 + n^2 \ | \ n \in \{ 0, 1, 2, 3, 4 \} \} $ \\
		\solucion: \\
		
		% Inciso --- (7) ---
		\item $ \{ \frac{1}{n^2 + n} \ | \ n \ \text{ es un entero positivo impar y } \ n \in \{ 1,2,3,5,7 \} \} $ \\
		\solucion: \\
		
	\end{enumerate}

	% Inciso --- 1.1.4 ---
	\item Determine cuáles de los siguientes conjuntos son iguales.
	\begin{flushleft}
		\begin{tabular}{ll}
			1) $ A = \{ c, e, r, o \} $ & 2) $ B = \{ 1, 2, 3 \} $ \\
			3) $ C = \{ a, r, o, m, a \} $ & 4) $ D = \{ 5, 2, 3, 4, 5 \} $ \\
			5) $ E = \{ x \in \mathbb{N} \ | \ 1 < x \leq 5 \} $ & 6) $ F = \{ x \in \mathbb{Z} \ | \ x^2 + 1 = 0 \} $ \\
			7) $ G = \{ 1, 2, 2 \} $ & 8) $ H = \{ e, s, p, o, n, j, a \} $ \\
			9) $ I = \{ 1, 2, 2, 3 \} $ & 10) $ J = \{ x \in \mathbb{N} \ | \ x^2 + 2  = 3x \} $ \\
			11) $ K = \{ 0 \} $ & 12) $ L = \{ a, m, o, r \} $ \\
			13) $ M = \{ j, a, p, o, n, e, s \} $ & 14) $ N = \emptyset $
		\end{tabular}
	\end{flushleft}
	\solucion: \\
	
	% Inciso --- 1.1.5 ---
	\item Si $ A = \{ x \in \mathbb{Z} \ | \ \exists y \in \mathbb{Z}(x = 2y) \} $ y $ B = \{ a \in \mathbb{Z} \ | \ \exists b, c \in \mathbb{Z} (a = 6b + 10c) \} $, demuestre que $ A = B $. \\
	\solucion: \\
	
\end{enumerate}

\section{Ejercicios sección 1.2}

\section{Ejercicios sección 1.3}

\section{Ejercicios sección 1.4}

\section{Ejercicios sección 1.5}

\section{Ejercicios sección 1.6}

\section{Ejercicios sección 1.7}

\section{Ejercicios sección 1.8}